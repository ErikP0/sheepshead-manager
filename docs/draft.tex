\documentclass[10pt,a4paper]{article}
\usepackage[utf8]{inputenc}
\usepackage{amsmath}
\usepackage{amsfonts}
\usepackage{amssymb}
\usepackage{graphicx}
\usepackage{tabularx}
\usepackage[left=1.00cm, right=1.00cm, top=1.00cm, bottom=1.00cm]{geometry}
\begin{document}
	\title{Entwurf Schafkopf-Manager}
	\section{Beschreibung}
	Der Schafkopf-Manager ist eine App für Android, die zum Zählen/ Punkte verwalten verwendet wird. Der Benutzer kann für mehrere Schafkopf-Spiele die Ergebnisse (wer hat gewonnen, welches Spiel, Einsatz, etc) eintragen. Ein Spielergebnis wird mit dem bisherigen Spielstand verrechnet. Der Benutzer kann sich den jetzigen Spielstand bzw. Spiel-Verlauf anzeigen lassen, (in späteren Versionen) Statistiken, Diagramme anzeigen lassen und den Spielverlauf in mehrere Formate (z.B. Excel, PDF) exportieren und per Mail versenden.
	
	\section{Fortschrittsplanung}
	Der Entwicklungsfortschritt wird in Meilensteine aufgeteilt. Bis Meilenstein II soll die erste funktionierende Version entstehen.\\
	\\
	\begin{tabular}{|p{3cm}||p{7cm}|p{7cm}|}
		\hline
		Meilenstein & Ziel & Maßnahmen, TODOs etc\\
		\hline
		\centering I\newline Einzelnes Spiel eintragen& 
		Der Benutzer kann ein Ergebnis eintragen 
		& Logik: Spielergebnis sowie Zwischenzustände als Klassen, Klassen für Spieler/Spielart/Einsatz...\\
		&& GUI: Auswahl-Activity\\
		\hline
		\centering II\newline Mehrere Spiele eintragen& Der Benutzer kann eine Sitzung erstellen, Mitspieler und Einsatz eintragen. Der Benutzer kann sich den derzeitigen Spielstand anzeigen lassen. Der Benutzer kann ein neues Spielergebnis hinzufügen, welches den Spielstand verändert. &
		Logik: Spielergebnisse müssen in einen Spielstand eingefügt werden, Klasse für Sitzung\\
		&&GUI: Sitzung-Erstellen-Activity, Spielstand-Anzeigen-Activity\\
		\hline
		\centering III\newline Spielstand merken & Wird die App geschlossen (nicht nur minimiert) wird das laufende Spiel inklusive des Spielstands gespeichert und bei Neustart evtl. wieder geladen.&
		Logik: Serialisierung von Sitzung und Spielstand\\
		&&GUI: Auswahl-Dialog Neues Spiel oder vorhandenes Spiel verwenden\\
		\hline
		\centering IV\newline Einfacher Export Excel & Der Benutzer kann den Spielverlauf des laufenden Spiels als Exceltabelle exportieren und per Mail verschicken&Logik: Export-Funktion z.B. nach .csv\\
		&&GUI: evtl. Export-Menü, Übergang zum Mail-Client\\
		\hline
		\centering V\newline Einfaches Diagramm&
		Der Benutzer kann ein Verlaufsdiagramm (x-Achse einzelne Spielergebnisse, y-Achse Geld der Mitspieler) auf Basis der aktuellen Daten des Spiels anzeigen lassen&
		Logik: Spielverlauf für Diagrammzeichnen vorbereiten\newline
		Evtl. Library?\\
		&&GUI: Diagramm-Activity zum Zeichnen des Diagramms\\
		\hline
	\end{tabular}
	
	\section{Weitere mögliche Features}
	\begin{itemize}
		\item Der Benutzer kann zusätzliche Informationen über das Spielergebnis eintragen, z.B. angespielte Sau/Farbe im Solo etc.
	\end{itemize}
\end{document}